\documentclass[12pt]{article}

\usepackage{amsmath,amssymb,amsthm} % flere matematikkommandoer
\usepackage[utf8]{inputenc} % æøå
\usepackage[T1]{fontenc} % mere æøå
\usepackage[danish]{babel} % orddeling
\usepackage{verbatim} % så man kan skrive ren tekst
\usepackage[all]{xy} % den sidste (avancerede)       formel i dokumentet
\usepackage[margin=1.0 in]{geometry}
%\addtolength{\topmargin}{.25 in}
\usepackage{graphics}
\usepackage{qtree}
\usepackage{tikz}
\usetikzlibrary{arrows}

\newcommand{\R}{\mathbb{R}}
\newcommand{\C}{\mathbb{C}}
\newcommand{\N}{\mathbb{N}}
\newcommand{\Z}{\mathbb{Z}}
\newcommand{\Q}{\mathbb{Q}}

\newcommand{\og}{\wedge}

\title{Noter til aflevering 1}
\author{Helena Bach}

\begin{document}
\maketitle

\section*{Kap 4}
\subsubsection*{1}
\textit{ Beskrive funktionelle / ikke-funktionelle krav til kursussystemet. Tag udgangspunkt
i ovenstående beskrivelse, men supplér gerne med jeres egen fortolkning af systemet.(OOSE side 119-121, 140-141, 160-161).
}\\\\
\textbf{Funktionelle krav:}\\
- projekt- og kursussystem hvor hvert kursus har sin egen side.\\
- Man kan blive tilmeldt som studerende eller underviser (m.fl.)\\
- muligt at gøre information om kursets afvikling tilgængelig
- uploade kursusmateriale i form af dokumenter\\
- gøre det muligt af aflevere opgaver\\
- Give feedback på opgaver\\
- Lade folk genaflevere opgaver\\
- En form for organisering af øvehold og grupper\\
- Mulighed for diskussion iblandt ovenstående\\
- Det skal være muligt at få en slags kvittering for sin aflevering.\\
- Der må gerne være en form for historik for afleveringer.\\\\
 \textbf{Ikke-funktionelle krav:}\\
- Kursussystemet skal bruges af studerende på alle KU’s fakulteter\\
- må der gerne tages højde for spidsbelastninger, og der må gerne være en strategi for når systemet overbelastes, så studerende kan aflevere deres opgaver.\\
- Kursussystemet bygger op imod større databaser af kurser og indskrevne studerende som bliver administreret i fakulteternes administrationer.\\
- Kursussystemet skal tematiseres så den indeholder fakultetets farver og logo.\\
\subsubsection*{2}
\textit{Beskrive roller/aktører i systemet (indenfor og udenfor problemområdet) og deres opgaver.
(OOSE side 9-10, 124-126, 149-152)}
\textbf{Roller}:\\
System adminstrator.\\
- dele rettigheder ud.\\
- 
Studerende\\
underviser\\
Observatør\\
Hjælpelærer\\

\textbf{Aktører}\\
- en aktør kan være de forskellige roller den samme person kan have. Oftes ser man på konkrete personer her.\\
Fx. Shine der både er studerende og underviser.
\subsubsection*{3}
\textit{Beskrive 1-2 scenarier med fokus på eksamen, tilmelding, aflevering eller diskussion
m.fl. Et eksempel som involverer flere use cases kunne være opdagelsen af en fejl i en
eksamensopgave. (OOSE side 126-128)}\\
en \textit{use case} er de forskellige cases der kan være. fx. en studende afleverer en opgave\\
\textit{Scenerie} er de forskellige udfald der kan være for en givet \textit{use case}. fx. studerende trykker på 'upload', men trykker ikke 'send'. En studerende trykker send, min oploader ikke nogen fil. studerende oploader og sender. studerende overskider frist. osv osv. 
\section*{Kap 5}
\end{document} 
