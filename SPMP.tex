\documentclass[12pt]{article}

\usepackage{amsmath,amssymb,amsthm}
\usepackage[T1]{fontenc}
\usepackage{graphics}
\usepackage{qtree}
\usepackage{tikz}
\usepackage[utf8]{inputenc} % æøå
\usepackage[T1]{fontenc} % mere æøå
\usepackage[danish]{babel} % orddeling
\usepackage{verbatim} % så man kan skrive ren tekst
\usepackage[all]{xy} % den sidste (avancerede) formel i dokumentet
\usepackage{fullpage} % mindre margin

\title{PKSU - Props 2.0} 
\author{Louise Knudsen, Helena Bach, David Pedersen}
\date{\today}

\newcommand{\R}{\mathbb{R}}
\newcommand{\C}{\mathbb{C}}
\newcommand{\N}{\mathbb{N}}
\newcommand{\Z}{\mathbb{Z}}
\newcommand{\Q}{\mathbb{Q}}

\newcommand{\og}{\wedge}

\begin{document}
\maketitle
\section{Problem definition}
\textit{The Royal Danish Theatre's props department uses a locally installed database-system to search through all the props and productions corresponding OPSTILLINGS- OG KØRELISTER. The system was developed more than 20 years ago by their own Claus Nepper Fakkenberg. Claus alone holds the responsibility for maintaining the system, as he is the only one who understands the source-code, which will soon be a problem as he is now retiring. \\
Due to the age of the system, some functionalities are no longer needed, and there are new requirements that are not implemented, most importantly a need to use the system outside of the office.} \\\\
The problem can be described with the following points:
\begin{itemize}
\item The current database system was developed in 1989, and lacks a lot of wanted functionalities.
\item The developer, Claus, is the only one who can modify and support the system.
\item Claus is retiring.
\end{itemize}
To analyze the problems defined above and develop a system definition we use the FACTOR Criterion.
\subsection{The FACTOR Criterion}
\begin{description}
  \item[Functionality:] Keeping track of the props and performances, and help in running these. As well as support the administration. 
  \item[Application domain:] The Royal Danish Theater.
  \item[Conditions:] The system should be usable by multiple user at a time, wherever there are access to the internet. 
  \item[Technology:] The system will be developed on standard laptops, and should be usable on standard PC's and tablets and supported by every OS. 
  \item[Object:] Props, pictures and employees in the props department. 
  \item[Responsibility:] Searching and administrative tool.
\end{description}
\subsection{System definition}
A web-based database-system of Det Kongelige Teaters props department. The system should primarily be a searching tool used to prepare and run performances and search for props in current and previous productions by the assistant stage managers, and secondly be used for budget monitoring, search for supplier information as well as keeping track of all performance information, by the chairman of the props department. The system should be usable for people with greatly variable computer experience.
\subsection{Requirements}
We have had meetings with different people and have some sort of idea of the requirements but only april 9th these will be formalised more specifically. \\
Acceptance criteria - A minimum of requirements that our system should meet.
\subsection{Constraints}
- Login, they handle that somehow...\\
- Should be able to talk with a media database that they have. We make sure they can talk to each other.
\subsection{Application domain}
The whole theatre.
\subsection{Solution domain}
The props department. \\
Cumulus, their media database. Only a limited part of this will be interfacing with our application though. \\
Most likely some sort of mirror of their data. 
\subsection{Deliverables}
- PHP code \\
- MySQL database \\
- Instructions in the form of meetings of papers explaining the system.\\
- Probably also some support during the initial start up phase.
\section{Initial Software Project Management Plan}
\subsection{Overview of the project}
\subsubsection{Project summary}
\textbf{Work product} \\
- mySQL database \\
- PHP code for the web-interface. \\
- Assignments describing the work process\\
\textbf{Schedule}\\
On April 9th we are scheduled to meet with Martin and Mikkel from the theatre, to finalize the requirements elicitation and sign the project agreement. We will thereafter be able to put together a more detailed time-schedule.\\
On June 23rd we have our final deadline, and will hopefully deliver the final product to the theatre. \\
\textbf{Participants}\\
Developers: David Pedersen, Helena Bach, Louise Knudsen \\
Project managers: David Pedersen, Helena Bach, Louise Knudsen \\
Client: The Royal Danish Theatre. \\
\textbf{Tasks}\\
noget med opgaver WORK BREAKDOWN STRUCTURE
\subsubsection{Evolution of the plan}
All changes to the project management plan must be agreed to by all participants before they are implemented. All changes should be documented in the, to the developers, shared log, and in time become a part of the final SPMP.
\subsection{References}
All artifacts will conform to the theatres standards. \\
noget med PHP og mySQL??
\subsection{Definitions}
noget med SKILT
\subsection{Project Organization}
\subsubsection{External interfaces}
The props database and appertaining web-interface will be managed and developed by David Pedersen, Helena Bach and Louise Knudsen. \\
The photographs of the props and productions, along with the login system and server connection will be handled by the XX, Martin Thaarup Larsen. \\
All managers will be in continuous contact with the chairman of the props department, Mikkel Rasmus Theut, as well as Martin Thaarup Larsen. \\
\subsubsection{Internal structure}
Skal vi have den her med? det samme som partisipants?
\subsubsection{Roles and responsibilities}
Noget med opdeling af arbejdet. noget med skill matrix: David 1 i web, Helena og Louise 1 i DB?? Alle vil være indover alle opgaver, men David har ansvaret for web, Helena og Louise for DB?????
\subsection{Managerial process plans}
\subsubsection{Start-up plan}
The total developments time is estimated to be 15 week, counting from 13/03/2014, where first meeting with the client were held. \\
Noget med roller i løbet af processen. \\
All necessary hardware are available, as well are the software since MySQL and PHP both are open-source.
\subsubsection{Work plan}
WEEK 1-5:
\begin{itemize}
\item Initial meeting with the chairman of the props department.
\item More technically oriented meeting with the XX and the chairman of the props department.
\item Meeting with head of the furniture subdivision.
Prepare the requirements elicitation (kravs spek.???) and the Project Agreement.
\item Meeting with he XX and the chairman of the props department - signing of the requirements elicitation (kravs spek.???) and the Project Agreement, and project kick-off!
\end{itemize}
WEEK 6-7: 
\begin{itemize}
\item Focus on making the SQL-script.
\end{itemize}
WEEK 8-11: 
\begin{itemize}
\item Finish SQL.
\item Start design of the web-application.
\item Begin PHP-coding and testing.
\end{itemize}
WEEK 12-15:
\begin{itemize}
\item Finish PHP-coding and testing.
\item Final documentation. 
\end{itemize}
\subsubsection{Control plan}
Noget i forhold til Roles/responsibilities, med at David har ansvaret for Web, og Helena og Louise har for DB. Vi har fælles ansvar for dokumentation. Dette sker med intern kontakt mindst 2 gange ugenligt.
\subsubsection{Risk management plan}
The risk factors and the tracking mechanisms are as follows. \\\\
As one of the goals of the project is to achieve more functionality than the existing system, there should in the testing-process be compared results between the two. \\\\
There should in the designing-process be a great amount of communication with the client to ensure an as user-friendly interface as possible, as the system mainly will be used by people with limited computer-experience. \\\\
There is a slim chance of hardware failure, in which case each developer is responsible for there own computer, and the client holds the responsibility for the server. \\\\
Each developer is responsible for continuous testing of their assigned subsystem(s) and jointly cross-testing between these.
\subsubsection{Closeout plan}
noget ????  
\subsection{Techinical process plans}
noget med kap 15...
\subsubsection{Process model}
\subsubsection{Methods, tools, and techniques}
\subsubsection{Infrastructure}
\subsubsection{Product acceptance plan}

\subsection{Supporting process plans}
\subsubsection{Configuration management plan}
Git and GitHub will be used in all aspects of the project to ensure version control.
\subsubsection{Verification and validation plan}
noget med test ift. 6'eren.
\subsubsection{Documentation plan}
måske noget med loggen, og PHP-Doc?
\subsubsection{Quality assurance plan}
noget???
\subsubsection{Reviews and audits}
??
\subsubsection{Problem resolution plan}
\subsubsection{Subcontractor management plan}
\subsubsection{Process improvement plan}
\subsection{Additional plans}
\subsubsection{Presentation}
In connection with the delivery of the final product, the system and its functionalities will be a presented to the chairman of the props department. \\
As one of the purposes of developing the new system, is to make it possible for the XX to keep the system up to date himself, we will in addition to the presentation, be going through code with the XX (Martin).
\subsubsection{Support}
During the initialisation phase support will be offered free of charge.
\section{Initial software architecture} 
\section{Project Agreement definition}

\end{document}
